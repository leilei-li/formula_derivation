\documentclass[11pt]{report}
\usepackage{ctex}
\usepackage{amstext}
\usepackage{amsmath}
\usepackage{amsfonts}
\title{机器学习经典算法推导}
\author{黎雷蕾}
\begin{document}
\maketitle
\tableofcontents
\chapter{代数基础}
\section{损失函数}
\subsection{0-1损失(Zero-one Loss)}
最简单的损失函数,如果预测值$\hat{y}_i$与目标值$y_i$不相等,那么为1,否则为0.
\begin{equation}
	\ell(y_i,\hat{y}_i)=
	\begin{cases}
		&1,\quad y_i\neq\hat{y}_i;\\
		&0,\quad y_i=\hat{y}_i.
	\end{cases}
\end{equation}
\subsection{感知损失(Perceptron Loss)}
用来改进0-1损失中判定较为严格的问题:
\begin{equation}
	\ell(y_i,\hat{y}_i)=
	\begin{cases}
		&1,\quad |y_i-\hat{y}_i|>t\\
		&0,\quad |y_i-\hat{y}_i|\leq t.
	\end{cases}
\end{equation}
\subsection{Hinge Loss}
Hinge Loss可以用来解决SVM只能够的间隔最大化问题。
\begin{equation}
\begin{split}
	\ell(y_i,\hat{y}_i)&=\max\{0,1-y_i\cdot\hat{y}_i\},\\
	&y_i\in\{-1,+1\}, \quad \hat{y}_i\in[-1,+1].
\end{split}
\end{equation}
\subsection{绝对值误差}
常用回归中:
\begin{equation}
	\ell(y_i,\hat{y}_i)=|y_i-\hat{y}_i|
\end{equation}
\subsection{均方误差}
常用于回归中:
\begin{equation}
	\ell(y_i,\hat{y}_i)=\frac{1}{n}\sum_{i=0}^n(y_i-\hat{y}_i)^2,
\end{equation}
\subsection{交叉熵}
神经网络、逻辑回归中常用的损失函数,二分类问题可写作:
\begin{equation}
	\begin{split}
		\ell(y_i,\hat{y}_i)&=y_i\log\hat{y}_i+(1-y_i)\log(1-\hat{y}_i)\\
		&y_i\in\{0,1\}
	\end{split}
\end{equation}
多分类问题可写作:
\begin{equation}
	\ell(y_i,\hat{y}_i)=-\sum_{i=0}^ny_i\log(\hat{y}_i)
\end{equation}
\subsection{指数误差(Exponential)}
常用于boosting算法:
\begin{equation}
	\ell(y_i,\hat{y}_i)=\exp(-y_i\cdot\hat{y}_i)
\end{equation}

\section{数值优化算法}
\subsection{牛顿法(Newton's method)}
牛顿法是是一种在实数域和复数域上近似求解方程的方法。方法使用函数$f(x)$的泰勒级数的前面几项来寻找方程$f(x)=0$的根。也被称为切线法。
\par
将$f(x)=0$在$x_0$处展开成泰勒级数:
\begin{equation}
	f(x_0)\rightarrow\sum_{n=0}^\infty\frac{f^{(n)}(x_0)}{n!}(x-x_0)^n
\end{equation}
我们只取其线性部分,作为非线性方程的近似方程:
\begin{equation}
	f(x_0)+(x-x_0)f'(x_0)=0
\end{equation}
设$f'(x_0)\neq0$,则其解为:
\begin{equation}
	x=x_1=x_0-\frac{f(x_0)}{f'(x_0)}
\end{equation}
这个公式说明$f(x_1)$的值将会比$f(x_0)$更加接近$f(x)=0$,我们就可以用迭代法逼近:
\begin{equation}
	x_{n+1}=x_n-\frac{f(x_n)}{f'(x_n)}
\end{equation}
最好是用二阶导数:
\begin{equation}
	x_{n+1}=x_n-\frac{f'(x_n)}{f''(x_n)}
\end{equation}
通过迭代,上式必然会在$f(x)=0$处收敛。
\subsection{拟牛顿法(Quasi-Newton Methods)}
拟牛顿法(Quasi-Newton Methods)是求解非线性优化问题最有效的方法之一。
\subsubsection{海森矩阵(Hessian Matrix)}
海森矩阵是一个多元函数的二阶偏导数构成的方阵,描述了函数的局部曲率。
假设二元函数$f(x_1,x_2)$在$\textbf{X}^{(0)}(x_1^{(0)},x_2^{(0)})$点处的泰勒展开式为:
\begin{equation}
	\begin{split}
		&f(x_1,x_2)=\\
		&f(x_1^{(0)},x_2^{(0)})+\left.\frac{\partial f}{\partial x_1}\right|_{\textbf{X}^{(0)}}\Delta x_1+\left.\frac{\partial f}{\partial x_2}\right|_{\textbf{X}^{(0)}}\Delta x_2+\\
		&\frac{1}{2}\left[\left.\frac{\partial^2 f}{\partial x_1^2}\right|_{\textbf{X}^{(0)}}\Delta x_1^2+\left.2\frac{\partial^2 f}{\partial x_1\partial x_2}\right|_{\textbf{X}^{(0)}}\Delta x_1\Delta x_2+\left.\frac{\partial^2 f}{\partial x_2^2}\right|_{\textbf{X}^{(0)}}\Delta x_2^2\right]+\cdots
	\end{split}
\end{equation}
其中,$\Delta x_1=x_1-x_1^{(0)},\Delta x_2=x_2-x_2^{(0)}$.
\par
将上式写成矩阵形式:
\begin{equation}
	\begin{split}
		&f(\textbf{X})=\\
		&f(\textbf{X}^{(0)})+\left(\frac{\partial f}{\partial x_1},\frac{\partial f}{\partial x_2} \right)_{\textbf{X}^{(0)}}\left(\begin{matrix}
			\Delta x_1\\
			\Delta x_2
		\end{matrix} \right)+\\
		&\frac{1}{2}(\Delta x_1,\Delta x_2 )\left.\left(\begin{matrix}
			\frac{\partial^2 f}{\partial x_1^2} & \frac{\partial^2 f}{\partial x_1\partial x_2}\\
			\frac{\partial^2 f}{\partial x_2\partial x_1} & \frac{\partial^2 f}{\partial x_2^2}
		\end{matrix}  \right)\right|_{\textbf{X}^{(0)}}\left(\begin{matrix}
			\Delta x_1\\
			\Delta x_2
		\end{matrix} \right)+\cdots
	\end{split}
\end{equation}
即:
\begin{equation}
	f(\textbf{X})=f(\textbf{X}^{(0)})+\nabla f(\textbf{X}^{(0)})^T\Delta \textbf{X}+\frac{1}{2}\Delta\textbf{X}^TH(\textbf{X}^{(0)})\Delta\textbf{X}+\cdots
\end{equation}
其中:
\begin{equation}
	H(\textbf{X}^{(0)})=
	\left.\left(
	\begin{matrix}
		\frac{\partial^2 f}{\partial x_1^2} & \frac{\partial^2 f}{\partial x_1\partial x_2}\\
		\frac{\partial^2 f}{\partial x_2\partial x_1} & \frac{\partial^2 f}{\partial x_2^2}
	\end{matrix}
	\right)\right|_{\textbf{X}^{(0)}},\ \Delta\textbf{X}=
	\left(\begin{matrix}
			\Delta x_1\\
			\Delta x_2
	\end{matrix} \right)
\end{equation}
$H(\textbf{X}^{(0)})$是$f(x_1,x_2)$在$\textbf{X}^{(0)}$处的海森矩阵,由$f(x_1,x_2)$在$\textbf{X}^{(0)}$处的二阶偏导数组成。
\par
将海森矩阵扩展到$n$元函数,对应的梯度$\nabla f(\textbf{X}^{(0)})$和海森矩阵可以写作$H(\textbf{X}^{(0)})$:
\begin{equation}
	\nabla f(\textbf{X}^{(0)})=\left.\left[\frac{\partial f}{\partial x_1},\frac{\partial f}{\partial x_2},\cdots,\frac{\partial f}{\partial x_n}\right]\right|^T_{\textbf{X}^{(0)}}
\end{equation}
\begin{equation}
	H(\textbf{X}^{(0)})=
	\left[
	\begin{matrix}
		\frac{\partial^2 f}{\partial x_1^2} & \frac{\partial^2 f}{\partial x_1\partial x_2} & \cdots & \frac{\partial^2 f}{\partial x_1\partial x_n}\\
		\frac{\partial^2 f}{\partial x_2 \partial x_1} & \frac{\partial^2 f}{\partial x_2^2} & \cdots & \frac{\partial^2 f}{\partial x_2\partial x_n}\\
		\vdots & \vdots & \ddots &\vdots\\
		\frac{\partial^2 f}{\partial x_n \partial x_1} & \frac{\partial^2 f}{\partial x_n \partial x_2} & \cdots & \frac{\partial^2 f}{\partial x_n^2}
	\end{matrix}
	\right]_{\textbf{X}^{(0)}}
\end{equation}
\subsubsection{拟牛顿法思想}
在牛顿法的迭代中,需要计算海森矩阵的逆矩阵$H^{-1}$,这个计算十分复杂,所以考虑到用一个$n$阶矩阵$G_k=G(x^{(k)})$来近似代替$H^{-1}_k=H^{-1}(x^{(k)})$,这就是拟牛顿法的基本思想了。
\par
假设$g_k=g(x^{(k)})=\nabla f(x^{k})$是$f(x)$的梯度向量在点$x^{(k)}$的值。那么牛顿法的更新公式就可以写作:
\begin{equation}
	x^{(k+1)}=x^{(k)}-H_k^{-1}g_k
\end{equation}
拟牛顿法的公式可以写作:
\begin{equation}
	x^{(k+1)}=x^{(k)}-G_kg_k
\end{equation}
\subsubsection{Davidon Fletcher Powell, DFP算法}
DFP选择$G_{k+1}$的方法是假设每一步迭代中矩阵$G_{k+1}$是由$G_k$加上两个附加项构成的:
\begin{equation}
	G_{k+1}=G_k+P_k+Q_k
\end{equation}
设$y_k=g_{k+1}-g_k$,$\delta_k=x^{(k+1)}-x^{(k)}$,那么可以求出$P_k,Q_k$:
\begin{equation}
	\begin{split}
		&P_k=\frac{\delta_k\delta_k^T}{\delta_k^T y_k},\\
		&Q_k=-\frac{G_ky_ky_k^TG_k^T}{y_k^TG_ky_k}
	\end{split}
\end{equation}
那么DFP算法中$G_{k+1}$的迭代公式可以写作:
\begin{equation}
	G_{k+1}=G_{k}+\frac{\delta_k\delta_k^T}{\delta_k^T y_k}-\frac{G_ky_ky_k^TG_k^T}{y_k^TG_ky_k}
\end{equation}
那么优化迭代公式可以写作:
\begin{equation}
	\begin{split}
		x^{(k+2)}&=x^{(k+1)}-G_{k+1}g_{k+1}\\
		&=x^{(k+1)}-\left(G_{k}+\frac{\delta_k\delta_k^T}{\delta_k^T y_k}-\frac{G_ky_ky_k^TG_k^T}{y_k^TG_ky_k}\right)g_{k+1}
	\end{split}
\end{equation}
\subsubsection{Broyden Fletcher Goldfarb Shanno, BFGS算法}
与DFP算法不同的是,BFGS采用一个容易求解逆矩阵的$B_k$去逼近海森矩阵本身$H$,而不是海森矩阵逆矩阵$H^{-1}$。
\par
那么BFGS的迭代公式可以写为:
\begin{equation}
	\begin{split}
		x^{(k+2)}&=x^{(k+1)}-B_{k+1}^{-1}g_{k+1}\\
		&=x^{(k+1)}-\left(B_{k}+\frac{y_ky_k^T}{y_k^T \delta_k}-\frac{B_k\delta_k\delta_k^TB_k^T}{\delta_k^TB_k\delta_k}\right)^{-1}g_{k+1}
	\end{split}
\end{equation}
\subsection{梯度下降(Gradient descent)}
\subsubsection{梯度}
所谓梯度,就是指函数变化最快的地方。对于一个函数$f(x,y)$,分别对于$x,y$求偏导,获得的梯度向量为$(\frac{\partial f}{\partial x},\frac{\partial f}{\partial y})^T$,简称为$grad\ f(x,y)$ 或者$\nabla f(x,y)$。梯度方向指的是沿着梯度向量$\nabla f(x,y)$的方向。
\subsubsection{梯度下降和梯度上升}
两者的实质是一样的,梯度下降取相反数就是梯度上升了。
\subsubsection{梯度下降的代数方法表示}
假设一个线性回归函数的表示公式为:
\begin{equation}
	h_\theta(x)=\sum_{i=0}^n\theta_i x_i
\end{equation}
损失函数取均方误差:
\begin{equation}
	J(\theta)=\frac{1}{n}\sum_{i=0}^n(h_\theta(x)-y_i)^2
\end{equation}
求出$J(\theta)$的梯度:
\begin{equation}
	\frac{\partial J(\theta)}{\partial \theta_i}=\frac{2}{n}\left(\sum_{i=0}^n\frac{\partial h_\theta(x)}{\partial \theta_i}-y_i\right)x_i^{(j)}
\end{equation}
那么更新的梯度表达式可以写作:
\begin{equation}
	\theta'_i=\theta_i-\alpha\frac{\partial J(\theta)}{\partial \theta_i}
\end{equation}
其中$\alpha\in[0,1]$称为学习步长,取值过大会造成震荡,太小会造成收敛过慢。
\subsubsection{梯度下降的矩阵方法表示}
假设一个线性回归函数的矩阵表示公式为:
\begin{equation}
	h_\theta(\textbf{x})=\textbf{X}\theta
\end{equation}
那么损失函数可以表示为:
\begin{equation}
	J(\theta)=\frac{1}{2}(\textbf{X}\theta-\textbf{Y})^T(\textbf{X}\theta-\textbf{Y})
\end{equation}
求出$J(\theta)$的梯度:
\begin{equation}
\begin{split}
	&\frac{\partial}{\partial\textbf{X}}(\textbf{X}\textbf{X}^T)=2\textbf{X}\\
	&\frac{\partial}{\partial\theta}(\textbf{X}\theta)=\textbf{X}^T\\
	&\frac{\partial}{\partial\theta}J(\theta)=\textbf{X}^T(\textbf{X}\theta-\textbf{Y})
\end{split}
\end{equation}
那么更新的梯度表达式可以写作:
\begin{equation}
	\theta'_i=\theta_i-\alpha\frac{\partial}{\partial \theta_i}J(\theta)
\end{equation}
\subsubsection{梯度下降参数调优}
\begin{enumerate}
	\item 步长$\alpha$:太大会造成震荡从而错过最优解,太小会造成迭代速度太慢。
	\item 参数初始值选择,梯度下降求出的是局部最优解,所以参数初始值不同求出的解也会不同,最好是正态分布随机生成。
	\item 归一化,可以加快迭代速度。
\end{enumerate}
\subsubsection{与其它优化算法比较}
梯度下降法和最小二乘法相比,梯度下降法需要选择步长,而最小二乘法不需要。梯度下降法是迭代求解,最小二乘法是计算解析解。如果样本量不算很大,且存在解析解,最小二乘法比起梯度下降法要有优势,计算速度很快。但是如果样本量很大,用最小二乘法由于需要求一个超级大的逆矩阵,这时就很难或者很慢才能求解解析解了,使用迭代的梯度下降法比较有优势。
\par
梯度下降法和牛顿法/拟牛顿法相比,两者都是迭代求解,不过梯度下降法是梯度求解,而牛顿法/拟牛顿法是用二阶的海森矩阵的逆矩阵或伪逆矩阵求解。相对而言,使用牛顿法/拟牛顿法收敛更快。但是每次迭代的时间比梯度下降法长。
\subsection{Momentum}
Momentum借鉴了物理学中动量的思想,通过积累之前的动量$m_{t-1}$来加速当前的梯度。设$\mu$是动量因子,通常设为0.9或其近似值:
\begin{equation}
	\begin{split}
		&m_t=\mu\cdot m_{t-1}+\alpha\nabla J(\theta)\\
		&\theta'_t=\theta_t-m_t
	\end{split}
\end{equation}
特点:
\begin{itemize}
	\item 接近局部最优解时,$\mu$的存在可以使更新幅度变大,从而跳出局部最优;
	\item 当前后两次梯度方向一致时,可以加速收敛,当前后两次梯度方向不一致时,可以减少震荡。
\end{itemize}
\subsection{Nesterov}
Nesterov是基于Momentum的算法,可以在梯度更新时对当前梯度进行一个矫正,避免前进太快,同时提高灵敏度:
\begin{equation}
	\begin{split}
		&m_t=\mu\cdot m_{t-1}+\alpha\nabla J(\theta-\mu\cdot m_{t-1})\\
		&\theta'_t=\theta_t-m_t
	\end{split}
\end{equation}
\subsection{Adagrad}
Adagrad对学习速率进行了一个约束。
设梯度$g_t=\nabla_\theta J(\theta)$,那么:
\begin{equation}
	\begin{split}
		& n_t=n_{t-1}+(g_t)^2\\
		& \theta_t=\theta_{t-1}-\frac{\eta}{\sqrt{n_t+\epsilon}}\cdot g_t=\theta_{t-1}-\frac{\eta}{\sqrt{\sum_{i=1}^tg_r^2+\epsilon}}\cdot g_t
	\end{split}
\end{equation}
其中$\eta$是一个全局学习率,$\epsilon$是一个常数来保证分母不为0。
\begin{itemize}
	\item 优点:
		\begin{itemize}
			\item 前期$n_t$较小的时候,梯度的系数较大,能够放大梯度;
			\item 后期$n_t$较大的时候,梯度的系数较小,能够缩小梯度;
		\end{itemize}
	\item 缺点:
		\begin{itemize}
			\item 中后期梯度系数会逐渐趋于0,可以使得训练提前结束,这可能会导致无法取得最优值。
		\end{itemize}
\end{itemize}
\subsection{Adadelta}
针对Adagrad会提前结束的问题,Adadelta只累计固定大小的项,并且也不直接存储这些项,仅仅计算对应的近似平均值。
\begin{equation}
\begin{split}
	& g_t=\nabla_\theta J(\theta)\\
	& n_t=v\cdot n_{t-1}+(1-v)\cdot g_t^2\\
	& \theta_t=\theta_{t-1}-\frac{\eta}{\sqrt{n_t+\epsilon}}\cdot g_t\\
	& E[g^2]_t=\rho\cdot E[g^2]_{t-1}+(1-\rho)\cdot g_t^2
\end{split}
\end{equation}
综上,系数项可以写为:
\begin{equation}
	-\frac{\sum_{i=1}^{t-1}\Delta\theta_i}{\sqrt{E[g^2]_t+\epsilon}}
\end{equation}
其中$E$代表期望,$v,\rho$是常数。
\par
特点:
\begin{itemize}
	\item 训练初期,加速效果很好;
	\item 训练后期,反复在局部最优解附近震荡,但是不会停止。
\end{itemize}
\subsection{RMSprop}
RMSprop是Adadelta的一个特例,即$\rho=0.5$。
\subsection{Adam}
Adam是一个非常好用的优化器,基本可以当成很多算法的初始优化器了。Adam本质是带有动量项的Adadelta。
\begin{equation}
	\begin{split}
		& m_t=\mu\cdot m_{t-1}+(1-\mu_t)\cdot g_t\\
		& n_t=v\cdot n_{t_1}+(1-v)\cdot g_t^2\\
		& \hat{m}_t=\frac{m_t}{1-\mu^t}\\
		& \hat{n}_t=\frac{n_t}{1-v^t}\\
		& \Delta\theta=-\frac{\hat{m}_t}{\sqrt{\hat{n}_t}+\epsilon}\cdot\eta
	\end{split}
\end{equation}
其中的参数初始设置:$\mu=0.9,v=0.999.\epsilon=10^{-8}$。该算法特点:
\begin{itemize}
	\item Adma参数比较平稳。
	\item 善于处理非平稳目标。
	\item 学习速率是自动计算出来的,不需要自己设置。
	\item 适用于大多数非凸优化问题。
\end{itemize}
\subsection{Adamax}
Adamax是Adam的一种变化,使得Adma的学习率边界范围更简单。
\begin{equation}
	\begin{split}
		& n_t=\max(v\cdot n_{t-1},|g_t|)\\
		& \Delta\theta_t=-\frac{\hat{m}_t}{n_t+\epsilon}\cdot \eta
	\end{split}
\end{equation}
\subsection{Nadam}
Nadam类似于带有Nexterov动量项的Adam算法.
\begin{equation}
	\begin{split}
		& \hat{g}_t=\frac{g_t}{1-\prod_{i=1}^t\mu_i}\\
		& m_t=\mu\cdot m_{t-1}+(1-\mu_t)\cdot g_t\\
		& \hat{m}_t=\frac{m_t}{1-\prod_{i=1}^{t+1}\mu_i}\\
		& n_t=v\cdot n_{t-1}+(1-v)\cdot g_t^2\\
		& \hat{n}_t=\frac{n_t}{1-v^t}\hat{m}_t=(1-\mu_t)\cdot\hat{g}_t+\mu_{t+1}\cdot\hat{m}_t\\
		& \Delta\theta_t=-\frac{\hat{m}_t}{\sqrt{\hat{n}_t}+\epsilon}\cdot\eta
	\end{split}
\end{equation}
一般而言,在使用带动量的RMSprop或Adam问题,使用Nadam可以取得更好的效果。
\section{拉格朗日乘子法}
拉格朗日乘子法就是求函数$f(x_1,x_2,\cdots)$在约束条件$g(x_1,x_2,\cdots)=0$下的极值的方法。
步骤如下:
\begin{enumerate}
	\item 设需要求极值的目标函数为$f(x_1,x_2,\cdots)$,限制条件为$g(x_1,x_2,\cdots)=0$;
	\item 引入$n$个拉格朗日参数$\lambda_i,i=1,2,\cdots,n$;
	\item 建立;拉格朗日函数
		\begin{equation}
			L(x_1,x_2,\cdots,x_n,\lambda_1,\lambda_2,\cdots,\lambda_n)=f(x_1,x_2,\cdots,x_n)+\sum_{i=1}^n\lambda_ig_i(x_1,x_2,\cdots,x_n)
		\end{equation}
	\item 对每一个参数求偏导,并令其为0,从而得到极值点:
		\begin{equation}
			\begin{split}
				& \frac{\partial L}{\partial x_i}=0\\
				& \frac{\partial L}{\partial \lambda_i}=0\\
				& i\in \{1,2,\cdots,n\}
			\end{split}
		\end{equation}
\end{enumerate}
\section{最小二乘法}
最小二乘估计法,又称最小平方法,是一种数学优化技术。它通过最小化误差的平方和寻找数据的最佳函数匹配。利用最小二乘估计法可以简便地求得未知的数据,并使得这些求得的数据与实际数据之间误差的平方和为最小。常用于线性回归模型。
\par
假设要回归的线性方程为$y=\beta_0+\beta_1x$,它的损失函数也是十分简单,就是假设$y_i$是真实值,$\hat{y}_i=\beta_0+\beta_1x$是估计值。那么损失函数可以写为:
\begin{equation}
	L=\sum_i^n(y_i-\hat{y}_i)^2=\sum_i^n(y_i-\beta_0-\beta_1x)^2
\end{equation}
要令$L$取最小值,分别对所有系数$\beta_0,\beta_1$进行求导:
\begin{equation}
	\begin{split}
		& \frac{\partial L}{\partial \beta_0}=2\sum_i^n(y_i-\beta_0-\beta_1x_i)(-1)=0\\
		& \frac{\partial L}{\partial \beta_1}=2\sum_i^n(y_i-\beta_0-\beta_1x_i)(-x_i)=0
	\end{split}
\end{equation}
经过整理我们就可以求出$\beta_0,\beta_1$:
\begin{equation}
	\begin{split}
		& \beta_0=\frac{\sum_i^nx_i^2\sum_i^ny_i-\sum_i^nx_i\sum_i^nx_iy_i}{n\sum_i^nx_i^2-(\sum_i^nx_i)^2}\\
		& \beta_1=\frac{n\sum_i^nx_iy_i-\sum_i^nx_i\sum_i^ny_i}{n\sum_i^nx_i^2-(\sum_i^nx_i)^2}
	\end{split}
\end{equation}
有了$\beta_0,\beta_1$两个参数,就可以回归线性方程了。
\chapter{BP算法}
假设样本输入为$x_i$,对应的标签为$y_i$。
\par
假设第$h$个隐藏层输入为:$\alpha_h=\sum_i v_{ih}x_i$,输出为:$b_h$。
\par
假设第$j$个输出层输入为:$\beta_j=\sum_j w_{hj}b_h$,输出为:$\hat{y}_j=f(\beta_j-\theta_j)$,其中$f(x)$为激活函数,这里取sigmoid,$\theta$为阈值。
\par
综上,我们可以建立从输入到输出的的联系:
\begin{equation}
\label{NN_process}
	\begin{cases}
		& \alpha_h=\sum_i v_{ih}x_i\\
		& b_h=f_1(\alpha_h-\gamma_h)\\
		& \beta_j=\sum_j w_{hj}b_h\\
		& \hat{y}_j=f_2(\beta_j-\theta_j)
	\end{cases}
\end{equation}
对于某个样本$k$,计算均方误差,为了方便,增加一个系数$\frac{1}{2}$:
\begin{equation}
	E_k=\frac{1}{2}\sum_j(\hat{y}_j^k-y_j^k)^2
\end{equation}
\par
我们进行随机梯度下降,满足:
\begin{equation}
\label{SGD}
	g(x+\Delta x)<g(x)
\end{equation}
根据泰勒展开式展开到一阶:
\begin{equation}
\begin{split}
	f(x+\Delta x)&\approx f(x)+\Delta xf'(x)+\frac{1}{2!}\Delta xf^{(2)}(x)+\cdots+\frac{1}{n!}\Delta xf^{(n)}(x)\\
	&\approx f(x)+\Delta xf'(x)
\end{split}
\end{equation}
那么公式\ref{SGD}可以转化为:
\begin{equation}
\begin{split}
	g(x+\Delta x)<g(x)&\Rightarrow g(x)+\Delta xg'(x)<g(x)\\
	&\Rightarrow \Delta xg'(x)<0
\end{split}
\end{equation}
我们只需让$\Delta xg'(x)$趋近于一个接近零的极小负数即可,引入学习速率$\eta$,由梯度下降的公式及偏导数的定义:
\begin{equation}
	v\leftarrow v+\Delta v
\end{equation}
\begin{equation}
\label{optimal_target}
	\Delta x =-\eta\frac{\partial L}{\partial x}
\end{equation}
其中,$L$就是我们定义的损失函数,$x$就是需要优化的参数了。
\section{$\Delta w$更新公式推导}
对于某个样本$k$,由公式\ref{optimal_target}我们可以推出$\Delta w$的优化公式:
\begin{equation}
\label{delta_w}
	\Delta w_{hj}=-\eta\frac{\partial E_k}{\partial w_{hj}}
\end{equation}
由公式\ref{NN_process}我们可以知道:$w$先影响$\beta$再影响$\hat{y}$最后影响$E$,这就构成了一个误差逆向传播链,那么由链式法则可以知道:
\begin{equation}
\label{w_123}
	\frac{\partial E_k}{\partial w_{hj}}=\frac{\partial E_k}{\partial \hat{y}_j^k}\cdot\frac{\partial \hat{y}_j^k}{\partial \beta_j}\cdot\frac{\partial \beta_j}{\partial w_{hj}}
\end{equation}
因为$\beta_j=\sum_j w_{hj}b_h$,那么:
\begin{equation}
\label{w_3}
\begin{split}
	\frac{\partial \beta_j}{\partial w_{hj}}&=\frac{\partial\left(\sum_j w_{hj}b_h \right)}{\partial w_{hj}}\\
		&=\frac{\partial\left(w_{h1}b_h+w_{h2}b_h+\cdots+w_{hj}b_h \right)}{\partial w_{hj}}\\
		&=b_h
\end{split}	
\end{equation}
\par
接下来推导$\frac{\partial E_k}{\partial \hat{y}_j^k}$:
\begin{equation}
\label{w_1}
	\begin{split}
		\frac{\partial E_k}{\partial \hat{y}_j^k}&=\frac{\left(\frac{1}{2}\sum_j(\hat{y}_j^k-y_j^k)^2 \right)}{\partial \hat{y}_j^k}\\
		&=\frac{\partial\left(\frac{1}{2}\left((\hat{y}_1^k-y_1^k)^2+(\hat{y}_2^k-y_2^k)^2+\cdots+(\hat{y}_j^k-y_j^k)^2\right) \right)}{\partial \hat{y}_j^k}\\
		&=\frac{\partial\left(\frac{1}{2} (\hat{y}_j^k-y_j^k)^2\right)}{\partial \hat{y}_j^k}\\
		&=\hat{y}_j^k-y_j^k
	\end{split}
\end{equation}
\par
往下继续推导$\frac{\partial \hat{y}_j^k}{\partial \beta_j}$,这里需要借用一个sigmoid函数的特殊性质:
\begin{equation}
\label{sigmoid_xingzhi}
	f'(x)=f(x)(1-f(x))
\end{equation}
借用sigmoid函数的性质,我们进行如下推导:
\begin{equation}
\label{w_2}
	\begin{split}
		\frac{\partial \hat{y}_j^k}{\partial \beta_j}&=\frac{\partial f(\beta_j^k-\theta_j)}{\partial \beta_j}\\
		&=f'(\beta_j^k-\theta_j)\\
		&=f(\beta_j^k-\theta_j)(1-f(\beta_j^k-\theta_j))\\
		&=\hat{y}_j^k(1-\hat{y}_j^k)
	\end{split}
\end{equation}
\par
我们简化偏导数的表达公式:
\begin{equation}
	\label{g_j}
	\begin{split}
		g_j&=\frac{\partial E_k}{\partial \hat{y}_j^k}\cdot\frac{\partial \hat{y}_j^k}{\partial \beta_j}\\
		&=(\hat{y}_j^k-y_j^k)\hat{y}_j^k(1-\hat{y}_j^k)
	\end{split}
\end{equation}
至此,综合公式\ref{w_123}、\ref{w_1}、\ref{w_2}、\ref{w_3}、\ref{g_j}我们可以求出$\Delta w_{hj}$的更新公式了。
\begin{equation}
\label{w_update}
	\begin{split}
		\Delta w_{hj}&=-\eta g_j b_h\\
		&=-\eta(\hat{y}_j^k-y_j^k)\hat{y}_j^k(1-\hat{y}_j^k)b_h
	\end{split}
\end{equation}
\par
这说明$w$的更新完全可以由输出$\hat{y}$,label $y$和上一层的输出$b_h$进行更新了。
\section{$\Delta \theta$更新公式推导}
同理,我们可以由公式\ref{optimal_target}来确定$\theta_j$	的更新公式:
\begin{equation}
	\begin{split}
		\Delta \theta_j &=-\eta\frac{\partial E_k}{\partial \theta_j}\\
		&=-\eta\frac{\partial\left(\frac{1}{2}\sum_j(\hat{y}_j^k-y_j^k)^2 \right)}{\partial \theta_j}\\
		&=-\eta\frac{\partial\left(\frac{1}{2}\sum_j(f(\beta_j^k-\theta_j)-y_j^k)^2 \right)}{\partial \theta_j}\\
		&=-\eta\left((f(\beta_j^k-\theta_j)-y_j^k)\cdot f'(\beta_j^k-\theta_j) \right)\\
		&=-\eta\left((\hat{y}_j^k-y_j^k)\cdot f(\beta_j^k-\theta_j)\cdot (1-f(\beta_j^k-\theta_j) )\right)\\
		&=-\eta\left((\hat{y}_j^k-y_j^k)\cdot \hat{y}_j^k \cdot (1-\hat{y}_j^k) \right)\\
		&=-\eta g_j
	\end{split}
\end{equation}
\section{$\Delta v$更新公式推导}
同公式\ref{delta_w}一致,结合链式法则,我们可以写出$\Delta v$的更新公式:
\begin{equation}
	\label{delta_v}
	\begin{split}
		\Delta v_{ih}&=-\eta\cdot\frac{\partial E_k}{\partial v_{ih}}\\
		&=-\eta\cdot\frac{\partial E_k}{\partial b_k}\cdot\frac{\partial b_k}{\partial \alpha_h}\cdot\frac{\partial\alpha_h}{\partial v_{ih}}\\
		&=-\eta\cdot\frac{\partial E_k}{\partial v_{ih}}\cdot\frac{\partial f(\alpha_h-\gamma_h)}{\partial \alpha_h}\cdot \frac{\partial \sum_i v_{ih}x_i}{\partial v_{ih}}\\
		&=-\eta\cdot\frac{\partial E_k}{\partial v_{ih}}\cdot f'(\alpha_h-\gamma_h) \cdot x_i\\
		&\approx -\eta\cdot \frac{\partial E_k}{\partial \hat{y}_j^k}\frac{\partial \hat{y}_j^k}{\partial\beta_j}\frac{\partial\beta_j}{\partial b_h}\cdot f(\alpha_h-\gamma_h)(1-f(\alpha_h-\gamma_h))\cdot x_i\\
		&=-\eta\cdot g_j\frac{\partial\sum_j w_{hj}b_h}{\partial b_h}\cdot b_h(1-b_h)\cdot x_i\\
		&=-\eta\cdot g_j\sum_j w_{hj}\cdot b_h(1-b_h)\cdot x_i
		=-\eta e_h x_i
	\end{split}
\end{equation}
其中$e_h=b_h(1-b_h)\sum_j w_{hj}g_j$.
\section{$\Delta \gamma_h$更新公式推导}
同理,我们可以结合链式法则,由公式\ref{delta_w}来确定$\gamma_h$的更新公式:
\begin{equation}
	\label{gamma_h}
	\begin{split}
		\Delta \gamma_h &=-\eta\cdot \frac{\partial E_k}{\partial \gamma_h}\\
		&=-\eta\cdot \frac{\partial E_k}{\partial b_h}\frac{\partial b_h}{\partial \alpha_h}\cdot \frac{\partial \alpha_h}{\partial b_h} \frac{\partial b_h}{\partial\gamma_h}\\
		&=-\eta\cdot e_h\cdot \frac{1}{\frac{\partial b_h}{\partial \alpha_h}}\cdot -1 \\
		&=\eta e_h 
	\end{split}
\end{equation}

\chapter{下一个算法}

































 
\end{document}