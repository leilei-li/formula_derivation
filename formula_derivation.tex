\documentclass[11pt]{report}
\usepackage{ctex}
\usepackage{amstext}
\usepackage{amsmath}
\usepackage{amsfonts}
\title{机器学习经典算法推导}
\author{黎雷蕾}
\begin{document}
\maketitle
\tableofcontents
\chapter{BP算法}
假设样本输入为$x_i$,对应的标签为$y_i$。
\par
假设第$h$个隐藏层输入为:$\alpha_h=\sum_i v_{ih}x_i$,输出为:$b_h$。
\par
假设第$j$个输出层输入为:$\beta_j=\sum_j w_{hj}b_h$,输出为:$\hat{y}_j=f(\beta_j-\theta_j)$,其中$f(x)$为激活函数,这里取sigmoid,$\theta$为阈值。
\par
综上,我们可以建立从输入到输出的的联系:
\begin{equation}
\label{NN_process}
	\begin{cases}
		& \alpha_h=\sum_i v_{ih}x_i\\
		& b_h=f_1(\alpha_h-\gamma_h)\\
		& \beta_j=\sum_j w_{hj}b_h\\
		& \hat{y}_j=f_2(\beta_j-\theta_j)
	\end{cases}
\end{equation}
对于某个样本$k$,计算均方误差,为了方便,增加一个系数$\frac{1}{2}$:
\begin{equation}
	E_k=\frac{1}{2}\sum_j(\hat{y}_j^k-y_j^k)^2
\end{equation}
\par
我们进行随机梯度下降,满足:
\begin{equation}
\label{SGD}
	g(x+\Delta x)<g(x)
\end{equation}
根据泰勒展开式展开到一阶:
\begin{equation}
\begin{split}
	f(x+\Delta x)&\approx f(x)+\Delta xf'(x)+\frac{1}{2!}\Delta xf^{(2)}(x)+\cdots+\frac{1}{n!}\Delta xf^{(n)}(x)\\
	&\approx f(x)+\Delta xf'(x)
\end{split}
\end{equation}
那么公式\ref{SGD}可以转化为:
\begin{equation}
\begin{split}
	g(x+\Delta x)<g(x)&\Rightarrow g(x)+\Delta xg'(x)<g(x)\\
	&\Rightarrow \Delta xg'(x)<0
\end{split}
\end{equation}
我们只需让$\Delta xg'(x)$趋近于一个接近零的极小负数即可,引入学习速率$\eta$,由梯度下降的公式及偏导数的定义:
\begin{equation}
	v\leftarrow v+\Delta v
\end{equation}
\begin{equation}
\label{optimal_target}
	\Delta x =-\eta\frac{\partial L}{\partial x}
\end{equation}
其中,$L$就是我们定义的损失函数,$x$就是需要优化的参数了。
\section{$\Delta w$更新公式推导}
对于某个样本$k$,由公式\ref{optimal_target}我们可以推出$\Delta w$的优化公式:
\begin{equation}
\label{delta_w}
	\Delta w_{hj}=-\eta\frac{\partial E_k}{\partial w_{hj}}
\end{equation}
由公式\ref{NN_process}我们可以知道:$w$先影响$\beta$再影响$\hat{y}$最后影响$E$,这就构成了一个误差逆向传播链,那么由链式法则可以知道:
\begin{equation}
\label{w_123}
	\frac{\partial E_k}{\partial w_{hj}}=\frac{\partial E_k}{\partial \hat{y}_j^k}\cdot\frac{\partial \hat{y}_j^k}{\partial \beta_j}\cdot\frac{\partial \beta_j}{\partial w_{hj}}
\end{equation}
因为$\beta_j=\sum_j w_{hj}b_h$,那么:
\begin{equation}
\label{w_3}
\begin{split}
	\frac{\partial \beta_j}{\partial w_{hj}}&=\frac{\partial\left(\sum_j w_{hj}b_h \right)}{\partial w_{hj}}\\
		&=\frac{\partial\left(w_{h1}b_h+w_{h2}b_h+\cdots+w_{hj}b_h \right)}{\partial w_{hj}}\\
		&=b_h
\end{split}	
\end{equation}
\par
接下来推导$\frac{\partial E_k}{\partial \hat{y}_j^k}$:
\begin{equation}
\label{w_1}
	\begin{split}
		\frac{\partial E_k}{\partial \hat{y}_j^k}&=\frac{\left(\frac{1}{2}\sum_j(\hat{y}_j^k-y_j^k)^2 \right)}{\partial \hat{y}_j^k}\\
		&=\frac{\partial\left(\frac{1}{2}\left((\hat{y}_1^k-y_1^k)^2+(\hat{y}_2^k-y_2^k)^2+\cdots+(\hat{y}_j^k-y_j^k)^2\right) \right)}{\partial \hat{y}_j^k}\\
		&=\frac{\partial\left(\frac{1}{2} (\hat{y}_j^k-y_j^k)^2\right)}{\partial \hat{y}_j^k}\\
		&=\hat{y}_j^k-y_j^k
	\end{split}
\end{equation}
\par
往下继续推导$\frac{\partial \hat{y}_j^k}{\partial \beta_j}$,这里需要借用一个sigmoid函数的特殊性质:
\begin{equation}
\label{sigmoid_xingzhi}
	f'(x)=f(x)(1-f(x))
\end{equation}
借用sigmoid函数的性质,我们进行如下推导:
\begin{equation}
\label{w_2}
	\begin{split}
		\frac{\partial \hat{y}_j^k}{\partial \beta_j}&=\frac{\partial f(\beta_j^k-\theta_j)}{\partial \beta_j}\\
		&=f'(\beta_j^k-\theta_j)\\
		&=f(\beta_j^k-\theta_j)(1-f(\beta_j^k-\theta_j))\\
		&=\hat{y}_j^k(1-\hat{y}_j^k)
	\end{split}
\end{equation}
\par
我们简化偏导数的表达公式:
\begin{equation}
	\label{g_j}
	\begin{split}
		g_j&=\frac{\partial E_k}{\partial \hat{y}_j^k}\cdot\frac{\partial \hat{y}_j^k}{\partial \beta_j}\\
		&=(\hat{y}_j^k-y_j^k)\hat{y}_j^k(1-\hat{y}_j^k)
	\end{split}
\end{equation}
至此,综合公式\ref{w_123}、\ref{w_1}、\ref{w_2}、\ref{w_3}、\ref{g_j}我们可以求出$\Delta w_{hj}$的更新公式了。
\begin{equation}
\label{w_update}
	\begin{split}
		\Delta w_{hj}&=-\eta g_j b_h\\
		&=-\eta(\hat{y}_j^k-y_j^k)\hat{y}_j^k(1-\hat{y}_j^k)b_h
	\end{split}
\end{equation}
\par
这说明$w$的更新完全可以由输出$\hat{y}$,label $y$和上一层的输出$b_h$进行更新了。
\section{$\Delta \theta$更新公式推导}
同理,我们可以由公式\ref{optimal_target}来确定$\theta_j$	的更新公式:
\begin{equation}
	\begin{split}
		\Delta \theta_j &=-\eta\frac{\partial E_k}{\partial \theta_j}\\
		&=-\eta\frac{\partial\left(\frac{1}{2}\sum_j(\hat{y}_j^k-y_j^k)^2 \right)}{\partial \theta_j}\\
		&=-\eta\frac{\partial\left(\frac{1}{2}\sum_j(f(\beta_j^k-\theta_j)-y_j^k)^2 \right)}{\partial \theta_j}\\
		&=-\eta\left((f(\beta_j^k-\theta_j)-y_j^k)\cdot f'(\beta_j^k-\theta_j) \right)\\
		&=-\eta\left((\hat{y}_j^k-y_j^k)\cdot f(\beta_j^k-\theta_j)\cdot (1-f(\beta_j^k-\theta_j) )\right)\\
		&=-\eta\left((\hat{y}_j^k-y_j^k)\cdot \hat{y}_j^k \cdot (1-\hat{y}_j^k) \right)\\
		&=-\eta g_j
	\end{split}
\end{equation}
\section{$\Delta v$更新公式推导}
同公式\ref{delta_w}一致,结合链式法则,我们可以写出$\Delta v$的更新公式:
\begin{equation}
	\label{delta_v}
	\begin{split}
		\Delta v_{ih}&=-\eta\cdot\frac{\partial E_k}{\partial v_{ih}}\\
		&=-\eta\cdot\frac{\partial E_k}{\partial b_k}\cdot\frac{\partial b_k}{\partial \alpha_h}\cdot\frac{\partial\alpha_h}{\partial v_{ih}}\\
		&=-\eta\cdot\frac{\partial E_k}{\partial v_{ih}}\cdot\frac{\partial f(\alpha_h-\gamma_h)}{\partial \alpha_h}\cdot \frac{\partial \sum_i v_{ih}x_i}{\partial v_{ih}}\\
		&=-\eta\cdot\frac{\partial E_k}{\partial v_{ih}}\cdot f'(\alpha_h-\gamma_h) \cdot x_i\\
		&\approx -\eta\cdot \frac{\partial E_k}{\partial \hat{y}_j^k}\frac{\partial \hat{y}_j^k}{\partial\beta_j}\frac{\partial\beta_j}{\partial b_h}\cdot f(\alpha_h-\gamma_h)(1-f(\alpha_h-\gamma_h))\cdot x_i\\
		&=-\eta\cdot g_j\frac{\partial\sum_j w_{hj}b_h}{\partial b_h}\cdot b_h(1-b_h)\cdot x_i\\
		&=-\eta\cdot g_j\sum_j w_{hj}\cdot b_h(1-b_h)\cdot x_i
		=-\eta e_h x_i
	\end{split}
\end{equation}
其中$e_h=b_h(1-b_h)\sum_j w_{hj}g_j$.
\section{$\Delta \gamma_h$更新公式推导}
同理,我们可以结合链式法则,由公式\ref{delta_w}来确定$\gamma_h$的更新公式:
\begin{equation}
	\label{gamma_h}
	\begin{split}
		\Delta \gamma_h &=-\eta\cdot \frac{\partial E_k}{\partial \gamma_h}\\
		&=-\eta\cdot \frac{\partial E_k}{\partial b_h}\frac{\partial b_h}{\partial \alpha_h}\cdot \frac{\partial \alpha_h}{\partial b_h} \frac{\partial b_h}{\partial\gamma_h}\\
		&=-\eta\cdot e_h\cdot \frac{1}{\frac{\partial b_h}{\partial \alpha_h}}\cdot -1 \\
		&=\eta e_h 
	\end{split}
\end{equation}

\chapter{下一个算法}

































 
\end{document}